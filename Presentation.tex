\documentclass{beamer}
\usepackage[utf8]{inputenc}
\usepackage{tikz}
\usepackage{caption}
\usepackage{subcaption}
\usepackage[T1]{fontenc}
\usetheme{Warsaw}
\title{OpenStreetMaps + Spatialite + Python}
\author{Michał Dulko}
\date{23 marca 2015}

\begin{document}

	\begin{frame}
		\titlepage
	\end{frame}

	\begin{frame}{Agenda}
		\setcounter{tocdepth}{1}
		\tableofcontents
	\end{frame}

	\section{Wstęp}
		\subsection{OpenStreetMap}
			\begin{frame}{\insertsubsectionhead}
				\begin{itemize}
					\item Od 2004 roku.
					\item Darmowe.
					\item Dostępne.
					\item Całkiem dokładne.
				\end{itemize}
			\end{frame}

		\subsection{Dane}
			\begin{frame}{\insertsubsectionhead}
				\begin{itemize}
					\item OSM udostępnia API.
					\item Planet.osm \pause(XML variant over 554GB uncompressed, 39.6GB bz2 compressed and 26.7GB PBF)
					\item „Extracts” (np. http://download.geofabrik.de/)
				\end{itemize}
			\end{frame}


		\subsection{Spatialite}
			\begin{frame}{\insertsubsectionhead}
				\begin{itemize}
					\item Rozwinięcie SQLite.
					\item OpenSource.
					\item Łatwo dostępne.
					\item Zintegrowane z OSM.
				\end{itemize}
			\end{frame}
		
		\subsection{Python}
			\begin{frame}{\insertsubsectionhead}
				„You tell me.”
			\end{frame}

	\section{Demo}
		\begin{frame}
			\begin{Huge}
				\centerline{Demo}
			\end{Huge}
		\end{frame}

	\begin{frame}
		\begin{Huge}
			\centerline{Thank you!}
		\end{Huge}
	\end{frame}

\end{document}
